\chapter{Introdução}

A acessibilidade digital tem se tornado um tema de crescente importância em um mundo cada vez mais conectado. O desenvolvimento tecnológico, nos últimos anos, tem despertado a atenção do mundo todo, especialmente quando o assunto trata-se de Inteligência Artificial (IA). Porém, mesmo em um mundo conectado, o desenvolvimento de tecnologias assistivas ainda caminha a passos lentos, levando anos para chegarem às pessoas com deficiência, especialmente aos deficientes visuais, que, por não terem visão ou terem, porém, limitada, são prejudicados no contato com recursos inovadores existentes atualmente. 

Segundo dados da \citeonline{WHO2023}, estima-se que pelo menos 2,2 bilhões de pessoas em todo o mundo vivenciam algum grau de deficiência visual, o que ressalta a necessidade de soluções tecnológicas específicas para esse público, visando a melhoria da qualidade de vida e, também, o contato com tecnologias inovadoras, como a IA. No contexto brasileiro, o Instituto Brasileiro de Geografia e Estatística (IBGE) aponta que mais de 6 milhões de pessoas têm algum nível de deficiência visual, o que evidencia a expressividade dessa população, reforçando o caráter urgente de iniciativas que promovam a inclusão. Embora diversos recursos digitais e físicos já existam para auxiliar pessoas cegas ou com baixa visão, ainda há lacunas quanto a ferramentas de reconhecimento de ambientes em tempo real por meio de dispositivos móveis.

Frente a esse cenário, o presente trabalho propõe o desenvolvimento de um aplicativo multiplataforma, capaz de capturar imagens do ambiente e descrevê-las em áudio por meio de técnicas de IA, promovendo a inclusão tecnológica e, além disso, melhorando a qualidade de vida de deficientes visuais. De acordo com \citeonline{silvaneto2024tecnologias}, a IA surge como um meio promissor para ampliar o acesso ao conhecimento, eliminar barreiras e melhorar a experiência acadêmica de grupos historicamente excluídos, como pessoas com deficiência, minorias étnicas e socioeconômicas. Nesse sentido, a proposta está alinhada às demandas atuais de inclusão e às oportunidades tecnológicas propiciadas pela evolução da IA, especialmente ao desenvolvimento dos modelos de linguagem de grande porte (do inglês, Large Language Models - LLMs), onde muitos são disponibilizados de forma gratuita, com o código aberto, em plataformas como o Hugging Face.

Desta forma, dada a crescente popularização dos smartphones e a melhoria contínua dos recursos de câmera embutidos nesses dispositivos, caminhos foram abertos para soluções inovadoras no campo da visão computacional. Não obstante, não podemos encarar a acessibilidade como um recurso adicional da tecnologia, mas como uma parte fundamental do desenvolvimento, visando garantir o acesso a qualquer grupo que a utilize. Assim, o desenvolvimento de aplicativos que traduzem informações visuais em descrições textuais e, posteriormente, em áudio, pode transformar a forma como pessoas com deficiência visual interagem com o ambiente ao seu redor.

A relevância deste estudo decorre do compromisso social de promover inclusão e autonomia a grupos historicamente sub-representados na evolução tecnológica. Além dos dados estatísticos já mencionados, há também uma motivação acadêmica, pois o uso de modelos de código livre de LLM para descrever imagens em tempo real é um tema ainda em consolidação na literatura científica, onde muitos estudos optam por usar modelos já renomados no mercado, porém proprietários. De acordo com \citeonline{lecun2015deeplearning}, o aumento da capacidade computacional, a disponibilidade de grandes bases de dados e o desenvolvimento de ferramentas de análise de dados impulsionam significativamente o progresso em redes neurais profundas, podendo contribuir para existirem modelos cada vez mais robustos. Todavia, ainda há desafios relacionados à latência, ao custo computacional e à efetiva acurácia das legendas geradas.

Assim, ao comparar diferentes modelos de linguagem pré-treinados (por exemplo, \lstinline{Qwen2.5-VL-7B-Instruct}, \lstinline{Llama-3.2-11B-Vision-Instruct} e o \lstinline{llava-v1.6-mistral-7b-hf}), este trabalho busca evidenciar pontos fortes e limitações de cada abordagem, contribuindo para a consolidação de uma prática fundamentada na seleção e no emprego de soluções de IA em projetos de acessibilidade. Desta forma, este trabalho tem como objetivo desenvolver um aplicativo móvel que possibilite a descrição em áudio de elementos do ambiente para pessoas com deficiência visual, valendo-se de um servidor para processamento de imagens por meio de LLMs de código livre. Assim, como resultado final, será desenvolvida uma tecnologia assistiva que poderá contribuir fortemente para a melhoria da qualidade de vida de deficientes visuais e, além disso, proporcionar o uso de tecnologias inovadoras da atualidade.

Para alcançar esse objetivo, este trabalho está detalhado em 5 capítulos, além desta introdução, que buscou apresentar resumidamente a problemática, além de discorrer sobre o objetivo geral do desenvolvimento.

No Capítulo \ref{cap:02}, que trata do Referencial Teórico, valeu-se de uma pesquisa detalhada acerca dos conceitos de acessibilidade, tecnologias assistivas, visão computacional e do funcionamento de modelos de linguagem de grande porte, além de explorar brevemente questões sobre \textit{Text-to-Speech}. Esta foi realizada por meio de ferramentas como o Periódicos Capes, Google Acadêmico e demais repositórios de trabalhos da área. 

Em seguida, no Capítulo \ref{cap:03}, é apresentada a Metodologia, onde os métodos, ferramentas e procedimentos adotados para o desenvolvimento do aplicativo, incluindo a estratégia de \textit{benchmark} dos diferentes modelos de IA, são detalhados. 

Já os Resultados e Discussões, que estão no Capítulo \ref{cap:04}, descrevem os resultados obtidos nos testes práticos e no \textit{benchmark}, analisando as métricas de desempenho e a usabilidade da solução. 

Por fim, a Conclusão, no Capítulo \ref{cap:05}, retoma os objetivos iniciais, faz considerações finais sobre o trabalho realizado e sugere futuros aprimoramentos, dando encerramento ao trabalho.

