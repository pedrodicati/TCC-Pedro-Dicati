\chapter{Conclusões} \label{cap:05}

Este trabalho teve como objetivo principal o desenvolvimento de um aplicativo capaz de auxiliar pessoas com deficiência visual na compreensão do ambiente ao seu redor, por meio de descrições em linguagem natural convertidas em áudio. Para atingir esse propósito, foi necessário enfrentar desafios técnicos relacionados à escolha e integração de LLMs, à latência do sistema, à usabilidade do aplicativo e à acessibilidade da solução como um todo.

Uma das maiores dificuldades enfrentadas foi o equilíbrio entre qualidade das descrições e desempenho em tempo real. Foram avaliados três LLMs ao longo do desenvolvimento, e essa comparação revelou-se essencial para a construção de uma solução eficiente e utilizável. A escolha do modelo \lstinline{Qwen2.5-VL-7B-Instruct} representou um ponto-chave, destacando-se pelas boas descrições geradas e pela menor latência observada, o que resultou em uma experiência mais fluida para o usuário.

O protótipo desenvolvido alcançou os objetivos propostos: é funcional, acessível, e demonstra a viabilidade de utilizar modelos \textit{open source} para finalidades assistivas. Além disso, a adoção de uma arquitetura modular fortalece a flexibilidade da solução, facilitando a manutenção e possíveis substituições de componentes. A utilização de tecnologias abertas também reforça o caráter social e evolutivo do projeto, permitindo seu aprimoramento contínuo em sinergia com a comunidade científica.

Para trabalhos futuros, identificam-se algumas direções promissoras:
\begin{itemize}
    \item A execução local do modelo diretamente em dispositivos móveis, reduzindo a dependência de conexão com a internet, especialmente importante para usuários em regiões com infraestrutura limitada;
    \item O uso de técnicas de compressão e otimização de modelos, como quantização e destilação, para reduzir custos computacionais e latência;
    \item A evolução do sistema de TTS, com vozes mais naturais e adaptativas;
    \item A ampliação da robustez do sistema frente a ambientes desafiadores, como iluminação ruim, múltiplos objetos ou cenários ruidosos.
\end{itemize}

Com isso, o trabalho não apenas cumpre sua proposta inicial, mas também estabelece uma base sólida para futuras pesquisas e avanços na área de tecnologias assistivas. Ao integrar LLMs de ponta em um contexto de acessibilidade, o projeto contribui diretamente para a inclusão digital de pessoas com deficiência visual, reforçando o potencial transformador da inteligência artificial quando aplicada com responsabilidade social.
