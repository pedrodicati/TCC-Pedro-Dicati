\chapter{Referencial Teórico}  \label{cap:03}

Nessa sessão você deve realizar o seu referencial teórico, que consiste na construção de ideias fundamentadas na área que você está estudando.
É nessa cessão que você irá apresentar os conceitos que estará estudando para poder construir sua revisão bibliográfica ou seu software.

Por exemplo:
Se meu estudo é sobre aplicativos de celular para a educação matemática, posso criar uma seção sobre os aplicativos, sobre seu histórico de uso, como são usados na educação, quais são os resultados promissores e desvantagens do seu uso.

Posteriormente, poderia criar um subtópico falando a respeito de como os aplicativos já criados estão atuando na área de informática, quais são as áreas de aplicação, o que esses aplicativos atendem ou deixam de atender na área em que atuam.

Mais adiante, posso criar um capítulo sobre as tecnologias utilizadas para concepção do projeto, sendo eles ambientes de programação, ambientes de documentação do sistema, projeções de testes.

Veja bem, todas essas áreas pedem muita leitura sobre o tema, para que seja realizada uma boa fundamentação teórica do assunto.

Não esqueça de referenciar suas imagens no texto para que o Overleaf possa montar o índice de imagens. O código da \autoref{fig:rotuloImagem} abaixo insere imagens. Ele está melhor explicado no tutorial de Latex.\\

% comando para inserir figura
\begin{figure}[!h]
    \centering
    \includegraphics[width=0.7\linewidth]{imagens/exemplo.png}
    \caption{Legenda da Imagem}
    \label{fig:rotuloImagem}
\end{figure}
\newpage 

Não esqueça de referenciar suas tabelas no texto para que o Overleaf possa montar o índice de tabelas. O código da \autoref{tab:tabela} abaixo insere uma tabela que você pode gerar utilizando o gerador de tabelas. (\href{https://www.tablesgenerator.com/}{Clique aqui para Acessar o Gerador de Tabelas}. Esse passo a passo está melhor explicado no tutorial de Latex.\\

% comando para inserir tabelas
\begin{table}[!h]
\centering
\begin{tabular}{l|l|l}
\hline
\multicolumn{1}{c|}{\textbf{Tabela}} & \multicolumn{1}{c|}{\textbf{Título 1}} & \multicolumn{1}{c}{\textbf{Título 2}} \\ \hline
Assunto 1                            & A                                      & C                                     \\ \hline
Assunto 2                            & B                                      & D                                     \\ \hline
\end{tabular}
\caption{Tabela}\label{tab:tabela}
\end{table}

