\chapter{Referencial Teórico}  \label{cap:02}

O referencial teórico tem como objetivo fundamentar este estudo a partir de conceitos e pesquisas já consolidadas na literatura científica. Por meio da revisão de trabalhos acadêmicos, artigos e documentos técnicos, esta seção apresenta as principais bases conceituais relacionadas ao desenvolvimento deste projeto.

Inicialmente, são abordados aspectos fundamentais sobre acessibilidade e deficiência visual, destacando a importância da tecnologia assistiva para promover inclusão social. Em seguida, discutem-se as principais tecnologias de apoio utilizadas atualmente, incluindo soluções baseadas em visão computacional e inteligência artificial. Também são explorados conceitos relacionados a LLMs e sua aplicação na descrição automática de imagens, além de métricas de avaliação da qualidade das descrições geradas.

Por fim, são detalhadas as tecnologias de reconhecimento e síntese de fala (Speech-to-Text e Text-to-Speech), essenciais para a interface acessível do sistema proposto. Essas discussões servirão como base para a construção e análise do modelo desenvolvido, garantindo alinhamento com os avanços mais recentes na área.

\section{Acessibilidade e Deficiência Visual}

A acessibilidade digital refere-se à capacidade de indivíduos, independentemente de suas habilidades ou deficiências, acessarem e interagirem com informações e serviços disponíveis no ambiente digital. \citeonline{Torres2002} destacam que a acessibilidade no espaço digital envolve a adaptação de conteúdos e interfaces para garantir que pessoas com diferentes tipos de deficiência possam utilizá-los de forma eficaz.

No contexto das pessoas com deficiência visual, a acessibilidade digital é particularmente desafiadora. Conforme destacam \citeonline{Torres2002}, “as barreiras arquitetônicas não são o maior obstáculo enfrentado pelas pessoas portadoras de deficiência. O maior obstáculo está no acesso à informação e, consequentemente, a aspectos importantes relacionados à informação, como a educação, o trabalho e o lazer”. Desta forma, é evidente que se faz necessário melhorias nas tecnologias existentes para a distribuição e acesso das pessoas deficientes, como dito por \citeonline{romeo2019}, que analisam o uso de diferentes tecnologias para acesso a conteúdos digitais por pessoas com deficiência visual e sugerem melhorias nas recomendações existentes para a concepção de \textit{websites} acessíveis. Deste modo, com a evolução não somente dos \textit{websites}, mas também de todas as tecnologias, a melhoria da acessibilidade para as pessoas deficientes seria nítida.

\subsection{Dados e Estatísticas sobre a População com Deficiência Visual}

Compreender a dimensão da população com deficiência visual é crucial para justificar a relevância de soluções tecnológicas acessíveis, para que, como exposto anteriormente, a acessibilidade em ambientes digitais passe por uma melhora significativa e contribua para o acesso de todos os grupos da sociedade. \citeonline{Castro2008} conduziram um estudo que descreve a prevalência e os fatores associados às deficiências visuais, auditivas e físicas no Brasil, constatando que 68\% do público entrevistado possuía algum tipo de deficiência visual, sendo a dificuldade de enxergar a principal deficiência referida e, como um dos principais fatores agravantes, o envelhecimento, conforme apresentado pelos autores. Fica claro, portanto, que este público merece atenção redobrada, dado que, segundo o \citeonline{ibgecenso2022}, em 2022 o índice de envelhecimento da população brasileira chegou a 80,0, indicando que há 80 pessoas idosas para cada 100 crianças de 0 a 14 anos, mas, comparado a 2010, o índice de envelhecimento era menor, correspondendo a 44,8, evidenciando um aumento de 78,5\%. Esses números corroboram para que a acessibilidade digital seja um ponto de discussão sério e fundamental para o desenvolvimento da inclusão digital.

Além disso, o IBGE tem desempenhado um papel fundamental na coleta e análise de dados relacionados à população com deficiência no Brasil. A produção e divulgação dessas informações são essenciais para embasar políticas públicas e iniciativas voltadas à inclusão social. A coleta de informações ocorre por meio de pesquisas como a Pesquisa Nacional de Saúde (PNS) de 2019, a Pesquisa Nacional por Amostra de Domicílios Contínua (PNAD Contínua) de 2022 e o Censo Demográfico, cada uma com metodologias e objetivos específicos. Segundo \citeonline{Botelho2024}, “as pesquisas conduzidas pelo IBGE adotam as recomendações do Grupo de Washington de Estatísticas sobre Deficiência, mas empregam questionários distintos, o que demanda atenção dos usuários desses dados”. Esse fator evidencia a complexidade na interpretação dos indicadores e destaca a necessidade de critérios padronizados para garantir a comparabilidade ao longo do tempo. O artigo também ressalta que as pessoas com dificuldades mais severas são as que enfrentam os maiores desafios no acesso à educação e ao mercado de trabalho, o que reforça a importância de políticas públicas baseadas em dados precisos. Tais informações são fundamentais não apenas para o desenvolvimento de políticas sociais, mas também para a criação de tecnologias assistivas que possam atender às demandas dessa parcela da população.

\subsection{Panorama de Leis e Normas}

A legislação brasileira tem avançado significativamente na promoção da inclusão e acessibilidade para pessoas com deficiência visual, estabelecendo diretrizes que garantem o acesso igualitário a diversos aspectos da vida em sociedade, incluindo a educação, o trabalho e o lazer. A Lei Brasileira de Inclusão da Pessoa com Deficiência (Lei nº 13.146/2015), também conhecida como Estatuto da Pessoa com Deficiência, representa um marco regulatório importante, pois estabelece diretrizes para garantir acessibilidade em diversas esferas, incluindo educação, transporte, comunicação e tecnologia. Segundo \citeonline{Bruno2019}, a política nacional de inclusão digital tem se mostrado essencial para eliminar barreiras atitudinais e tecnológicas, proporcionando autonomia e participação ativa na sociedade para pessoas com deficiência visual. A acessibilidade digital, nesse contexto, é reconhecida como um direito fundamental, assegurando que todos os cidadãos tenham acesso igualitário às informações e oportunidades disponíveis no ambiente digital.

Além disso, a Política Nacional de Educação Especial na Perspectiva da Educação Inclusiva \cite{Brasil2008} reforça o papel da tecnologia assistiva como um recurso essencial para a inclusão educacional de pessoas com deficiência visual. O decreto nº 7.611/2011, que regulamenta a educação especial, define o Atendimento Educacional Especializado (AEE) como um conjunto de recursos e estratégias pedagógicas voltadas para a eliminação de barreiras ao aprendizado, com a implementação de salas de recursos multifuncionais equipadas com tecnologia assistiva específica, como softwares de leitura de tela e impressoras em braille. Essas iniciativas, no entanto, ainda enfrentam desafios quanto à implementação efetiva em diferentes níveis de ensino, conforme destacado por \citeonline{CHILINGUE2024}, que aponta que a ausência de adequação em ambientes virtuais de aprendizagem compromete a inclusão digital plena de estudantes com deficiência visual. Assim, embora haja avanços legislativos e normativos, a acessibilidade digital e educacional ainda requer esforços contínuos para garantir que as políticas sejam aplicadas de forma abrangente e eficaz.

Esses marcos legais reforçam a importância de desenvolver tecnologias assistivas que promovam a acessibilidade digital, garantindo que pessoas com deficiência visual possam exercer plenamente seus direitos e participar ativamente da sociedade.

\section{Tecnologias de Apoio}

As tecnologias assistivas desempenham um papel essencial na promoção da inclusão de pessoas com deficiência visual, permitindo-lhes superar barreiras e interagir de maneira mais autônoma com o mundo ao seu redor. Como visto anteriormente, aproximadamente 2,2 bilhões de pessoas em todo o mundo possuem algum grau de deficiência visual, sendo que uma parcela significativa enfrenta desafios na locomoção, acesso à informação e comunicação \cite{WHO2023}. A introdução de tecnologias assistivas têm proporcionado avanços significativos, melhorando a qualidade de vida e promovendo a equidade de acesso em diversos contextos, como a educação e o mercado de trabalho.

\subsection{Tecnologias Assistivas e sua Contribuição}

O termo Tecnologia Assistiva é utilizado para identificar todo o arsenal de recursos e serviços que contribuem para proporcionar ou ampliar habilidades funcionais de pessoas com deficiência e, consequentemente, promover vida independente e inclusão \cite{bersch2024}. Segundo \citeonline{silvaneto2024tecnologias}, a adoção dessas tecnologias tem sido crucial para garantir a acessibilidade digital, proporcionando autonomia no uso de computadores e dispositivos móveis por meio de recursos como leitores de tela e aplicativos de reconhecimento de imagem.

A utilização dessas ferramentas contribui significativamente para a inclusão de pessoas com deficiência visual em ambientes educacionais, profissionais e sociais. Conforme destacado em um estudo de \citeonline{BORGES2021}, o uso de softwares assistivos tem por finalidade eliminar as barreiras à plena participação e à vida funcional para as pessoas com deficiência, incapacidades e mobilidade reduzida, objetivando uma maior autonomia e qualidade de vida. Portanto, pode-se notar a importância destes dispositivos atualmente, principalmente dos \textit{smartphones} e outros dispositivos populares, dado o fácil acesso de toda a população, inclusive da parcela desfavorecida visualmente. Ademais, a inclusão digital e social dessas pessoas é fortalecida por iniciativas que combinam políticas públicas e o avanço tecnológico, promovendo um ambiente mais inclusivo e acessível, contribuindo ainda mais para o acesso à informação deste grupo.

\subsection{Aplicações Existentes}

Atualmente, diversas aplicações tecnológicas foram desenvolvidas para atender às necessidades das pessoas com deficiência visual, proporcionando maior autonomia em atividades do cotidiano. Essas tecnologias englobam desde soluções simples, como leitores de tela, até dispositivos mais avançados, como bengalas eletrônicas equipadas com sensores ultrassônicos. A seguir, são apresentados alguns exemplos de tecnologias assistivas que vêm sendo amplamente utilizadas:

\begin{enumerate}
    \item \textbf{Leitores de Tela:} Softwares que convertem texto digital em áudio, permitindo que os usuários acessem conteúdos online, documentos e aplicativos. Estudos demonstram que leitores de tela são ferramentas fundamentais para inclusão digital, proporcionando acesso equitativo à informação \cite{brilli2024}. Exemplos incluem:
        \begin{enumerate}
            \item NVDA (NonVisual Desktop Access): Software de código aberto, amplamente utilizado, compatível com múltiplos sistemas operacionais;
            \item JAWS (Job Access With Speech): Um dos leitores de tela mais completos, com suporte a diversas funcionalidades avançadas.
        \end{enumerate}
    \item \textbf{Aplicativos de Descrição de Imagens:} Aplicativos baseados em inteligência artificial capazes de analisar imagens e fornecer descrições detalhadas por meio de síntese de voz. Essas soluções ajudam os usuários a identificar objetos, reconhecer pessoas e entender contextos visuais. Segundo \citeonline{silvaneto2024tecnologias}, tais aplicativos têm demonstrado impacto significativo na autonomia de pessoas com deficiência visual. Exemplos notáveis:
        \begin{enumerate}
            \item Seeing AI (Microsoft): “Aplicativo móvel que fornece recursos de leitura de texto, moeda, produto, reconhecimento facial e descrição de cena” \cite{Dognin2022};
            \item Envision AI: Aplicativo que permite capturar imagens e obter descrições precisas em áudio.
        \end{enumerate}
    \item \textbf{Bengalas Eletrônicas:} As bengalas eletrônicas utilizam sensores de proximidade e \textit{feedback} hápticos para ajudar os usuários a detectar obstáculos no caminho. As localizações bem estudadas dos sensores utilizados permitem a locomoção segura e confortável do usuário, já que toda a cena à sua frente, do topo ao chão, é interpretada \cite{AmmarBouhamed2012}. Dispositivos como:
        \begin{enumerate}
            \item WeWALK: Uma bengala equipada com sensores ultrassônicos e integração com assistentes virtuais;
            \item UltraCane: Tecnologia baseada em sensores de eco para navegação segura em ambientes urbanos.
        \end{enumerate}
    \item \textbf{Softwares de OCR (Reconhecimento Óptico de Caracteres):} Conforme proposto por \citeonline{Sonth2017}, são aplicações capazes de realizar a conversão eletrônica de imagens em texto codificado por máquina, podendo, posteriormente, aplicar técnicas de síntese de voz para o apoio às pessoas com desafios visuais. Essas ferramentas são amplamente utilizadas para leitura de documentos físicos, proporcionando maior independência aos usuários. Exemplos incluem:
        \begin{enumerate}
            \item Google Lens: Reconhece e traduz textos a partir de imagens capturadas com a câmera do \textit{smartphone}.
            \item KNFB Reader: “[...] Mecanismo de OCR que funciona em um telefone celular e permite que uma pessoa com deficiência visual leia texto impresso de uma imagem tirada pela câmera” \cite{wang2010}.
        \end{enumerate}
    \item \textbf{Dispositivos Vestíveis Inteligentes:} A nova geração de tecnologias assistivas inclui dispositivos vestíveis que combinam sensores e inteligência artificial para fornecer informações contextuais sobre o ambiente. Estudos recentes mostram, conforme apontado por \citeonline{brilli2024} que esses dispositivos oferecem uma experiência mais intuitiva e personalizada, ampliando a interação com o ambiente. Exemplos incluem:
        \begin{enumerate}
            \item AIris: De acordo com Brilli et al. (2024):
                \begin{quote}
                    Um dispositivo vestível alimentado por IA que fornece recursos de conscientização e interação ambiental para usuários com deficiência visual. O AIris combina uma câmera sofisticada montada em óculos com uma interface de processamento de linguagem natural, permitindo que os usuários recebam descrições auditivas em tempo real de seus arredores.
                \end{quote}
            \item Envision Glasses: Óculos equipados com reconhecimento de texto e objetos para fornecer feedback auditivo detalhado.
        \end{enumerate}
\end{enumerate}


A \autoref{tab:tecnologiasassistivas} apresenta um resumo de algumas das principais tecnologias assistivas disponíveis para pessoas com deficiência visual, destacando suas funcionalidades e exemplos de aplicação.


\begin{table}[h!]
\centering
\caption{Tecnologias assistivas para deficientes visuais}
\label{tab:tecnologiasassistivas}
\resizebox{\columnwidth}{!}{%
\begin{tabular}{@{}lllll@{}}
\toprule
\textbf{Tecnologia Assistiva} &
  \multicolumn{1}{c}{\textbf{Descrição}} &
  \multicolumn{1}{c}{\textbf{Exemplos}} &
   &
   \\ \midrule
Leitores de Tela &
  Convertam texto digital em áudio ou braille, permitindo acesso a conteúdos online &
  NVDA, JAWS &
   &
   \\
Aplicativos de Descrição de Imagem &
  Utilizam IA para descrever imagens e identificar objetos em tempo real &
  Seeing AI, Envision AI &
   &
   \\
Bengalas Eletrônicas &
  Ajudam na navegação com sensores para detectar obstáculos e fornecer feedback tátil/auditivo &
  WeWALK, UltraCane &
   &
   \\
Softwares de OCR &
  Transformam imagens de textos impressos em texto digital para leitura em voz alta. &
  Google Lens, KNFB Reader &
   &
   \\
Dispositivos Vestíveis Inteligentes &
  Equipamentos que combinam sensores e inteligência artificial para fornecer informações contextuais sobre o ambiente. &
  AIris, Envision Glasses &
   &
   \\ \bottomrule
\end{tabular}%
}
\end{table}











% Nessa sessão você deve realizar o seu referencial teórico, que consiste na construção de ideias fundamentadas na área que você está estudando.
% É nessa sessão que você irá apresentar os conceitos que estará estudando para poder construir sua revisão bibliográfica ou seu software.

% Por exemplo:
% Se meu estudo é sobre aplicativos de celular para a educação matemática, posso criar uma seção sobre os aplicativos, sobre seu histórico de uso, como são usados na educação, quais são os resultados promissores e desvantagens do seu uso.

% Posteriormente, poderia criar um subtópico falando a respeito de como os aplicativos já criados estão atuando na área de informática, quais são as áreas de aplicação, o que esses aplicativos atendem ou deixam de atender na área em que atuam.

% Mais adiante, posso criar um capítulo sobre as tecnologias utilizadas para concepção do projeto, sendo eles ambientes de programação, ambientes de documentação do sistema, projeções de testes.

% Veja bem, todas essas áreas pedem muita leitura sobre o tema, para que seja realizada uma boa fundamentação teórica do assunto.

% Não esqueça de referenciar suas imagens no texto para que o Overleaf possa montar o índice de imagens. O código da \autoref{fig:rotuloImagem} abaixo insere imagens. Ele está melhor explicado no tutorial de Latex.\\

% % comando para inserir figura
% \begin{figure}[!h]
%     \centering
%     \includegraphics[width=0.7\linewidth]{imagens/exemplo.png}
%     \caption{Legenda da Imagem}
%     \label{fig:rotuloImagem}
% \end{figure}
% \newpage 

% Não esqueça de referenciar suas tabelas no texto para que o Overleaf possa montar o índice de tabelas. O código da \autoref{tab:tabela} abaixo insere uma tabela que você pode gerar utilizando o gerador de tabelas. (\href{https://www.tablesgenerator.com/}{Clique aqui para Acessar o Gerador de Tabelas}. Esse passo a passo está melhor explicado no tutorial de Latex.\\

% % comando para inserir tabelas
% \begin{table}[!h]
% \centering
% \begin{tabular}{l|l|l}
% \hline
% \multicolumn{1}{c|}{\textbf{Tabela}} & \multicolumn{1}{c|}{\textbf{Título 1}} & \multicolumn{1}{c}{\textbf{Título 2}} \\ \hline
% Assunto 1                            & A                                      & C                                     \\ \hline
% Assunto 2                            & B                                      & D                                     \\ \hline
% \end{tabular}
% \caption{Tabela}\label{tab:tabela}
% \end{table}

