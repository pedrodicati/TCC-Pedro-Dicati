%------------------------------------------------------%
%------------------------------------------------------%
%    Modelo de Projeto de TCC - TADS - PI 1 ( IFMS-NA) % 
% NÃO ALTERE OU APAGUE ESSAS INFORMAÇÕES
% ELAS SÃO A CONFIGURAÇÃO DO SEU ARQUIVO.
%-------------------------------------------------------
%-------------------------------------------------------


\documentclass[
	12pt,				% tamanho da fonte
	openright,			% capítulos começam em pág ímpar 
	twoside,			% para impressão em verso e anverso. Oposto a oneside
	a4paper,			% tamanho do papel. 
	%normalfigtabnum,
	%pnumromarab,
	% Opções da classe abntex2
	chapter=TITLE,		% títulos de capítulos convertidos em letras maiúsculasa
	%section=TITLE,		% títulos de seções convertidos em letras maiúsculas
	%subsection=TITLE,	% títulos de subseções convertidos em letras maiúsculas
	%subsubsection=TITLE,% títulos de subsubseções convertidos em letras maiúsculas
	% Opções do pacote babel
	english,			% idioma adicional para hifenização
    spanish,			% idioma adicional para hifenização
	brazil,				% o último idioma é o principal do documento
]{abntex2}

% ----------------------------------------------------------------
%                           PACOTES
% ----------------------------------------------------------------

		

\usepackage[utf8]{inputenc}		% O pacote inputenc é usado para que seja possível escrever textos acentuados em determinado padrão de codificação. No caso, abnTEX2 utiliza a codificação UTF8. Consulte detalhes do pacote em <http://www.ctan.org/pkg/inputenc>.
\usepackage[T1]{fontenc}		% O pacote fontenc controla a codificação das fontes usadas para impressão do documento. Consulte detalhes do pacote em <http://www.ctan.org/pkg/fontenc>.
\usepackage{times}
\usepackage{lastpage}			% Usado pela Ficha catalográfica
\usepackage{indentfirst}		% Indenta o primeiro parágrafo de cada seção.
\usepackage{xcolor,colortbl}	% Controle das cores
\usepackage{graphicx}			% Inclusão de gráficos
\usepackage{microtype} 			% para melhorias de justificação
\usepackage{hyperref}           % pacote para adicionar links internos
\usepackage{subfig}
\usepackage{epigraph}
%\usepackage[authoryear,round,longnamesfirst]{natbib}
\usepackage{url}
\usepackage{placeins}
\usepackage{multirow}
\usepackage[figuresright]{rotating}
\usepackage{chemfig}
\usepackage{amsmath}
\usepackage{amssymb}
\usepackage{enumitem}
\usepackage{bigints}
\usepackage{listings}
\usepackage{etoolbox}
\usepackage[final]{pdfpages}
\usepackage{bigstrut}
\usepackage{acronym}
\usepackage{longtable}
\usepackage{verbatim}          % pacote para adicionar comentarios em bloco
\usepackage{hypernat}
%\usepackage{titlesec}          % para permitir a alteracao em titulos

%-----------------------------------------------------
%-----------------------------------------------------
%	FLOATS: TABLES, FIGURES E CAPTIONS: CONFIGURAÇÕES
%-----------------------------------------------------

\usepackage{tabularx} % Better tables
\setlength{\extrarowheight}{3pt} % Increase table row height
\newcommand{\tableheadline}[1]{\multicolumn{1}{c}{\spacedlowsmallcaps{#1}}}
\newcommand{\myfloatalign}{\centering} % To be used with each float for alignment
\usepackage{caption}
\captionsetup{format=hang,font=small}
\usepackage{subfig}

%----------------------------------------------------
%	LIST ENUMERATION
%----------------------------------------------------

\renewcommand{\labelenumii}{\theenumii}
\renewcommand{\theenumii}{\theenumi.\arabic{enumii}.}
% ---
% Pacotes adicionais, usados apenas no âmbito do Modelo Canônico do abnteX2
% ---
\usepackage{lipsum}				% para geração de dummy text
% ---

% ---
% Pacotes de citações
% ---
\usepackage[brazilian,hyperpageref]{backref}	 % Paginas com as citações na bibl
\usepackage[alf,abnt-emphasize=bf]{abntex2cite}  % Citações padrão ABNT

% ------------------------------------------------------
%             CONFIGURAÇÕES DOS PACOTES
% ------------------------------------------------------

% ---
% Configurações do pacote backref
%
% Para desativar, tire o comentário de \begin{comment} e \end{comment} das próximas linhas e comente a linha \usepackage[brazilian,hyperpageref]{backref}
% acima.
% ---

%\begin{comment}
% ---
% Configurações do pacote backref
% Usado sem a opção hyperpageref de backref
\renewcommand{\backrefpagesname}{}
%\renewcommand{\backrefpagesname}{Citado na(s) página(s):~}
% Texto padrão antes do número das páginas
\renewcommand{\backref}{}
% Define os textos da citação
 \renewcommand*{\backrefalt}[4]{
% 	\ifcase #1 %
% 	Nenhuma citação no texto.%
% 	\or
% 	Citado na página #2.%
% 	\else
% 	Citado #1 vezes nas páginas #2.%
% 	\fi}%
}
% ---
%\end{comment}


% listagens
\definecolor{corComentario}{RGB}{150,150,150}
\definecolor{corString}{RGB}{206,123,0}
\definecolor{corPalavraChave}{RGB}{0,0,230}

\lstset{
	numbers=left,
	stepnumber=1,
	firstnumber=1,
	numberstyle=\footnotesize,
	extendedchars=true,
	breaklines=true,
	lineskip=0pt,
	frame=tb,
	basicstyle=\ttfamily\footnotesize,
	showstringspaces=false,
	stringstyle=\color{corString},
	commentstyle=\color{corComentario},
	keywordstyle=\color{corPalavraChave}
}

\newcolumntype{Y}{>{\centering\arraybackslash}X}

\newcommand{\ano}[1]{\def \oano {#1}}
\newcommand{\imprimirano}{\oano}

\newcommand{\mes}[1]{\def \omes {#1}}
\newcommand{\imprimirmes}{\omes}

\newcommand{\subtitulo}[1]{\def \osubtitulo {#1}}
\newcommand{\imprimirsubtitulo}{\osubtitulo}

\newcommand{\area}[1]{\def \aarea {#1}}
\newcommand{\imprimirarea}{\aarea}

\renewcommand{\coorientador}[1]{\def \ocoorientador {#1}}
\renewcommand{\imprimircoorientador}{\ocoorientador}

\newcommand{\grau}[1]{\def \ograu {#1}}
\newcommand{\imprimirgrau}{\ograu}

\newcommand{\curso}[1]{\def \ocurso {#1}}
\newcommand{\imprimircurso}{\ocurso}

\usepackage{edicoes}

%--------------------------------------------
%----------------Não Alterar-----------------
\curso{TECNOLOGIA EM ANÁLISE E DESENVOLVIMENTO DE SISTEMAS}
\grau{Tecnólogo em Análise e Desenvolvimento de Sistemas}
\tipotrabalho{Trabalho de Conclusão de Curso}
\local{NOVA ANDRADINA - MS}
\instituicao{%
	INSTITUTO FEDERAL DE EDUCAÇÃO, CIÊNCIA E TECNOLOGIA DE MATO GROSSO DO SUL
	CÂMPUS NOVA ANDRADINA
}
\preambulo{\imprimirtipotrabalho\ apresentado ao Instituto Federal de Educação, Ciência e Tecnologia de Mato Grosso do Sul – Câmpus Nova Andradina – como um dos requisitos para a obtenção do título de \imprimirgrau.\\
}

% ---

% ---
% Configurações de aparência do PDF final
% ---

% alterando o aspecto da cor azul
\definecolor{blue}{RGB}{41,5,195}

% informações do PDF
\makeatletter
\hypersetup{
	%pagebackref=true,
	pdftitle={\@title}, 
	pdfauthor={\@author},
	pdfsubject={\imprimirpreambulo},
	pdfcreator={Nome Completo},
	pdfkeywords={Palavra chave 1}{Palavra chave 2}{Palavra chave 3}{Palavra chave n}, 
	colorlinks=true,       		% false: boxed links; true: colored links
	linkcolor=black,          	% color of internal links
	citecolor=black,       		% color of links to bibliography
	filecolor=black,      		% color of file links
	urlcolor=blue,
	bookmarksdepth=4
}
\makeatother
% --- 

% ---
% Comandos do autor
% ---
% para ajustar o tamanho da fonte no cabeçalho

% Agradecimentos
\renewenvironment{agradecimentos}[1][\agradecimentosname]{%
    %\ABNTEXchapterfont{\large{#1}}
   \pretextualchapter{\large{#1}}
  }{\PRIVATEclearpageifneeded}
% ---

\renewenvironment{resumo}[1][\resumoname]{%
   \PRIVATEbookmarkthis{#1}
   \renewcommand{\abstractnamefont}{\chaptitlefont}
   \renewcommand{\abstractname}{\ABNTEXchapterupperifneeded{\large{#1}}}
   \begin{abstract}
  }{\end{abstract}\PRIVATEclearpageifneeded}

% comando para inserir autor e ano
\newcommand{\citeauthorandyear}[1]{\citeauthoronline{#1} (\citeyear{#1})}

% ---
% Novo list of (listings) para Quadros
% ---

\newcommand{\quadroname}{Quadro}
\newcommand{\listofquadrosname}{Lista de Quadros}

\newfloat[chapter]{quadro}{loq}{\quadroname}
\newlistof{listofquadros}{loq}{\listofquadrosname}
\newlistentry{quadro}{loq}{0}

% configurações para atender às regras da ABNT
\setfloatadjustment{quadro}{\centering}
\counterwithout{quadro}{chapter}
\renewcommand{\cftquadroname}{\quadroname\space} 
\renewcommand*{\cftquadroaftersnum}{\hfill--\hfill}

% Configuração de posicionamento padrão:
\setfloatlocations{quadro}{hbtp}

% ---
% Fontes padroes de part, chapter, section, subsection e subsubsection
\renewcommand{\ABNTEXchapterfont}{\large\bfseries}
\renewcommand{\ABNTEXchapterfontsize}{\large}

\renewcommand{\ABNTEXpartfont}{\normalfont}
\renewcommand{\ABNTEXpartfontsize}{\ABNTEXchapterfontsize}

\renewcommand{\ABNTEXsectionfont}{\normalfont\MakeUppercase}
\renewcommand{\ABNTEXsectionfontsize}{\Large}

\renewcommand{\ABNTEXsubsectionfont}{\normalfont\bfseries}
\renewcommand{\ABNTEXsubsectionfontsize}{\normalsize}

\renewcommand{\ABNTEXsubsubsectionfont}{\normalfont\bfseries}
\renewcommand{\ABNTEXsubsubsectionfontsize}{\normalsize\bfseries}

\renewcommand{\ABNTEXsubsubsubsectionfont}{\normalfont}
\renewcommand{\ABNTEXsubsubsubsectionfontsize}{\normalsize}
% ---




% ------------------------------------------------------------
%           Espaçamentos entre linhas e parágrafos 
% ------------------------------------------------------------

% O tamanho do parágrafo é dado por:
\setlength{\parindent}{1.3cm}

% Controle do espaçamento entre um parágrafo e outro:
\setlength{\parskip}{0.2cm}  % tente também \onelineskip

% indicando o caminha das imagens para todo o documento.
\graphicspath{ {imagens/} }

% ---------------------------------------------------------
%                   Compila o indice
% ---------------------------------------------------------

% Você pode customizar o nível de divisões que o sumário pode listar com a macro \settocdepth{hnome da subdivisãoi}, sendo nome da subdivisão UM dos valores: chapter, part, section, subsection, subsubsection
\settocdepth{subsection}

\makeindex
% ---

%===========================================
%                 CABEÇALHO
%===========================================

%%criar um novo estilo de cabeçalhos e rodapés
\makepagestyle{meuestilo}
  %%cabeçalhos
  \makeevenhead{meuestilo} %%pagina par
     {\thepage}         % esquerda
     {}         % centro
     {} % direita
  \makeoddhead{meuestilo} %%pagina ímpar ou com oneside
     {} % esquerda
     {}         % centro
     {\thepage}         % direita
  %\makeheadrule{meuestilo}{\textwidth}{\normalrulethickness} %linha
  %% rodapé
  %\makeevenfoot{meuestilo}
     %{rodapé par à esquerda} %%pagina par
     %{centro \thepage}
     %{\thepage} 
  %\makeoddfoot{meuestilo} %%pagina ímpar ou com oneside
     %{rodapé ímpar/onside à esquerda}
     %{centro \thepage}
     %{\thepage}


%===========================================
%               FIM CABEÇALHO
%===========================================


% ------------------------------------------------------
%                 INÍCIO DO DOCUMENTO
% ------------------------------------------------------
\begin{document}

% Seleciona o idioma do documento (conforme pacotes do babel)
%\selectlanguage{english}
\selectlanguage{brazil}

%\pagestyle{plain}
% Retira espaço extra obsoleto entre as frases.
\frenchspacing 

% ---------------------------------------------------
%   INSERIR PÁGINAS DOS ELEMENTOS PRÉ-TEXTUAIS
% -----------------------------------------------------
% \pretextual

% ---
% CapA
%\imprimircapa
\begin{center}
	
	%\center
	\ABNTEXchapterfont\bfseries\large\textsc{\textbf{\imprimirinstituicao}}
	\vspace{1.0cm}
    
    \ABNTEXchapterfont\large\textsc{\textbf{\imprimircurso}}
	\vspace{3.5cm}
	
    \ABNTEXchapterfont\large\textsc{\textbf{\imprimirautor}}
	\vspace{3.5cm}
	
    \ABNTEXchapterfont\large\textsc{\textbf{\imprimirtitulo\ifdef{\osubtitulo}{:}{}}}
    
    \ifdef{\osubtitulo}{\ABNTEXchapterfont\large\textbf{\imprimirsubtitulo}}{}
	\vfill
	
	\SingleSpacing
	\large\textsc{\textbf{\imprimirlocal}}\\
	\large\textsc{\textbf{\imprimirano}}

	%\vspace*{2cm}
	
\end{center}

\cleardoublepage
% ---
% Folha de rosto
%\imprimirfolhaderosto*
\begin{center}
   	
   	\ABNTEXchapterfont\large\textsc{\textbf{\imprimirautor}}
   	\vspace{6.5cm}
   	
    \ABNTEXchapterfont\large\textsc{\textbf{\imprimirtitulo}}\ifdef{\osubtitulo}{:}{}
                           
    \ifdef{\osubtitulo}{\ABNTEXchapterfont\large\textbf{\imprimirsubtitulo}}{}
   	\vspace{1.5cm}
   	   	
   	\hspace{.5\textwidth}
   	\begin{minipage}{.4\textwidth}
   		\SingleSpacing
   		\footnotesize\imprimirpreambulo
   		
   		%\vspace{\onelineskip}
   		
   		\textbf{Orientador:} \imprimirorientador
   		
        \ifdef{\ocoorientador}{
     		\vspace{\onelineskip}
   		
    	\textbf{Coorientador:} \imprimircoorientador
        }{}
   		
   	\end{minipage}%
    \vfill
   	
   	\SingleSpacing
   	\normalsize\textsc{\textbf{\imprimirlocal}}\\
   	\normalsize\textsc{\textbf{\imprimirano}}
   	
   	%\vspace*{2cm}
   	
\end{center}
% ---
% Inserir a ficha catalográfica
% --- para ficha, retire o % da linha de baixo.
%\input{Pre_textual/FichaCatalografica}

% Inserir folha de aprovação
% --- para inserir a folha de aprovação retire o % da linha de baixo.
%%
% Este é um exemplo de folha de aprovação.
%
% A folha de aprovação final será inserida pelo Coordenador do Curso.
%

\begin{folhadeaprovacao}
	\includepdf{folhaAprovacao/folha_aprovacao}
\end{folhadeaprovacao}
% ---
% Dedicatória
% --- para inserir dedicatória, retire o % da linha de baixo.
% \input{Pre_textual/Dedicatoria}
% ---
% Agradecimentos
\begin{agradecimentos}
    Agradeço primeiramente a Deus, fonte de força, sabedoria e inspiração em todos os momentos dessa jornada.
    
    À minha amada Natália Farina, pelo carinho, paciência e apoio incondicional, fundamentais para que eu seguisse firme neste caminho.
    
    Ao meu orientador, \imprimirorientador, por sua orientação, paciência, dedicação e valiosas contribuições para a realização deste trabalho. 
    
    Ao Instituto Federal de Mato Grosso do Sul (IFMS), \textit{campus} Três Lagoas, especialmente ao professor Vladimir Píccolo Barcelos, por viabilizar o acesso ao servidor e contribuir diretamente para a concretização deste projeto.
    
    À minha família, que sempre esteve ao meu lado, oferecendo apoio e motivação para sempre seguir adiante. 
    
    E, por fim, aos meus colegas da faculdade e demais professores, que compartilharam comigo desafios, aprendizados e momentos inesquecíveis nesta caminhada acadêmica.
\end{agradecimentos}
% ---
% Epígrafe
% --- para inserir Epígrafe, retire o % da linha de baixo
% \begin{epigrafe}
	
	\vspace*{\fill}
	\epigraph{``\emph{A tecnologia move o mundo}''.}{Steve Jobs}
	
\end{epigrafe}
% ---
% Resumos
\begin{resumo}

%\noindent{SILVA, João da. \textbf{Título do trabalho de conclusão de curso em negrito}. Nº de fls. TCC (Trabalho de Conclusão de Curso). Instituto Federal de Mato Grosso do Sul – IFMS. Tecnologia em Análise e Desenvolvimento de Sistemas, Câmpus Nova Andradina, MS. 2018.}

%\setlength{\absparsep}{18pt} % ajusta o espaçamento dos parágrafos do resumo
%\vspace{1.5cm}
	
A acessibilidade digital é um desafio crucial em um mundo cada vez mais conectado, especialmente para pessoas com deficiência visual. Este trabalho apresenta o desenvolvimento de um aplicativo multiplataforma assistivo capaz de capturar imagens do ambiente e descrevê-las em áudio, utilizando modelos de linguagem de grande porte (LLMs) de código aberto. O aplicativo integra recursos de visão computacional e síntese de voz (Text-to-Speech - TTS), promovendo maior autonomia para usuários com deficiência visual. A metodologia incluiu o \textit{benchmark} de três modelos de IA (Qwen 2.5, llava v1.6 Mistral e Llama 3.2 Vision), avaliados com base em métricas de latência, qualidade textual (BERTScore e ROUGE-L) e testes práticos em cenários reais. Os resultados demonstraram que o modelo Qwen 2.5 apresentou o melhor equilíbrio entre precisão descritiva e desempenho em tempo real. O protótipo desenvolvido evidencia o potencial da IA para ampliar a inclusão digital, destacando-se como uma solução inovadora para a melhoria da qualidade de vida de pessoas com deficiência visual.

	\vspace{\onelineskip}
	
	\textbf{Palavras-chave}: Acessibilidade Digital, Tecnologia Assistiva, Modelos de Linguagem Multimodais, Visão Computacional, Inclusão Social.
	
\end{resumo}
\begin{resumo}[Abstract]
	
	\begin{otherlanguage*}{english}
		
	Digital accessibility is a crucial challenge in an increasingly connected world, especially for people with visual impairments. This paper presents the development of a cross-platform assistive application capable of capturing environmental images and describing them in audio using open-source large language models (LLMs). The application integrates computer vision and text-to-speech (TTS) technologies, promoting greater autonomy for visually impaired users. The methodology included benchmarking three AI models (Qwen 2.5, llava v1.6 Mistral e Llama 3.2 Vision), evaluated based on latency metrics, textual quality (BERTScore and ROUGE-L), and practical tests in real-world scenarios. The results showed that the Qwen 2.5 model achieved the best balance between descriptive accuracy and real-time performance. The developed prototype demonstrates the potential of AI to enhance digital inclusion, standing out as an innovative solution to improve the quality of life for visually impaired individuals.
	
    \textbf{Keywords}: Digital Accessibility, Assistive Technology, Multimodal Language Models, Computer Vision, Social Inclusion.
		
	\end{otherlanguage*}

\end{resumo} 
% ---
% inserir lista de ilustrações
% ---
\renewcommand{\listfigurename}{\large{Lista de figuras}}
\pdfbookmark[0]{\listfigurename}{lof}
\listoffigures*  % O * remove esse item do indice
\cleardoublepage
% ---
% inserir lista de tabelas
% ---
\renewcommand{\listtablename}{\large{Lista de Tabelas}}
\pdfbookmark[0]{\listtablename}{lot}
\listoftables*
\cleardoublepage
% ---
% ---
% inserir lista de quadros
% --- Para inserir lista de quadros, tire o % da linha de baixo
% \pdfbookmark[0]{\listofquadrosname}{loq}
% \listofquadros*
% \cleardoublepage
% ---
% inserir lista de abreviaturas e siglas
% --- para inserir lista de Abreviaturas e Siglas, retire o % da linha de baixo
% \begin{siglas}
	\item AAAAAAAAAA
\end{siglas}


%*******************************************************
%                Acronimos
%*******************************************************
    %\phantomsection 
    %\refstepcounter{dummy}
    %\pdfbookmark[1]{Acronyms}{acronyms}
    %\markboth{\spacedlowsmallcaps{Acronyms}}{\spacedlowsmallcaps{Acronyms}}
    %\chapter*{\large{Acrônimos}}
    %\begin{acronym}[UML]
        %\acro{XXX}{nome aqui}
    %\end{acronym} 
    
    %\cleardoublepage
% ---

% ---
% inserir lista de símbolos
% ---
% \input{Pre_textual/ListaSimbolos}
% ---

% ------------------------------------
% INSERIR SUMÁRIO
% -----------------------------------
\renewcommand{\contentsname}{\large{SUMÁRIO}}

\pdfbookmark[0]{\contentsname}{toc}
\tableofcontents*
\cleardoublepage
% ---

% ----------------------------------------------
%             ELEMENTOS TEXTUAIS
%           INSERIR NOVOS ARQUIVOS
% ----------------------------------------------
% cada um desses comandos a baixo montam uma seção de capítulos do nosso documentos na hora da compilação.
% note que eles foram chamados com a ordem que eles aparecerão no documento final. Ideal é que no momento da criação, sejam renomeados de maneira lógica, numerando o capítulo e a seção.

\textual
\pagestyle{meuestilo}
\chapter{Introdução}

A introdução é um capítulo tão importante quanto qualquer outro. Nele você deve apresentar o tema do seu trabalho, bem como a delimitação que será assumida e a problemática que será analisada por você. A introdução do TCC deve explicar o contexto em que o tema está inserido e deixar claro o motivo do assunto ser importante e relevante para sua área de pesquisa.

Dessa maneira, para escrever uma introdução que aborde esses aspectos, podemos seguir a seguinte estrutura:

% Esse ambiente Insere uma lista não Numerada
\begin{itemize}
    \item \textbf{Apresentação do Tema e do Contexto da área da Pesquisa:} Devemos pensar que o leitor não conhece nosso tema e área da pesquisa na qual estamos atuando, devemos então explicar brevemente para ele do que se trata nosso trabalho e a área na qual estamos estudando. Por exemplo, se nosso estudo propõe um aplicativo de celular para a educação, podemos começar a introdução falando do uso dos aplicativos de Smarthphone para aproveitamento de conteúdo.  
    
    \item \textbf{Delimitação do Tema:} Nesse momento, após o leitor conhecer o tema de maneira ampla, podemos delimitar o assunto somente a área de estudos que iremos assumir. Por exemplo, se fora dito que o trabalho irá atuar na área de desenvolvimento de aplicativos para a educação, nesse momento iremos mostrar que tipo de desenvolvimento será feito, quais ferramentas e ambientes serão utilizados e qual área ele atende, no caso, sendo educação especial ou uma matéria de estudo.
    
    \item \textbf{Problemática:} Nesse parágrafo você deve mostrar ao seu leitor qual é o problema que você pretende solucionar. Sendo uma hipótese de um dado assunto ou o desenvolvimento de um software, é aqui que você deve mostrar porque sua ideia é diferente de todas as outras para responder uma problemática em comum. Por exemplo, ao iniciar o trabalho, se minha hipótese é "Um aplicativo de celular ajuda os alunos a aprender melhor um conteúdo de matemática?". É nesse paragrafo que eu devo mostrar ao meu leitor como surgiu essa pergunta e quais são as possíveis respostas que eu pretendo descobrir.
    
    \item \textbf{Objetivos e Metodologia}: Por fim, você deve explanar rapidamente sobre os objetivos que deseja alcançar com todo o seu projeto e a Metodologia Proposta.
\end{itemize}

Se eu posso dar uma dica, fica mais confortável escrever a Introdução depois de ter desenvolvido os Objetivos e a Metodologia do Trabalho.
E lembre-se, só escreve bem sobre um assunto quem lê e estuda muito ele.



\input{Capitulos/Cap02-Objetivos}
\chapter{Referencial Teórico}  \label{cap:03}

Nessa sessão você deve realizar o seu referencial teórico, que consiste na construção de ideias fundamentadas na área que você está estudando.
É nessa cessão que você irá apresentar os conceitos que estará estudando para poder construir sua revisão bibliográfica ou seu software.

Por exemplo:
Se meu estudo é sobre aplicativos de celular para a educação matemática, posso criar uma seção sobre os aplicativos, sobre seu histórico de uso, como são usados na educação, quais são os resultados promissores e desvantagens do seu uso.

Posteriormente, poderia criar um subtópico falando a respeito de como os aplicativos já criados estão atuando na área de informática, quais são as áreas de aplicação, o que esses aplicativos atendem ou deixam de atender na área em que atuam.

Mais adiante, posso criar um capítulo sobre as tecnologias utilizadas para concepção do projeto, sendo eles ambientes de programação, ambientes de documentação do sistema, projeções de testes.

Veja bem, todas essas áreas pedem muita leitura sobre o tema, para que seja realizada uma boa fundamentação teórica do assunto.

Não esqueça de referenciar suas imagens no texto para que o Overleaf possa montar o índice de imagens. O código da \autoref{fig:rotuloImagem} abaixo insere imagens. Ele está melhor explicado no tutorial de Latex.\\

% comando para inserir figura
\begin{figure}[!h]
    \centering
    \includegraphics[width=0.7\linewidth]{imagens/exemplo.png}
    \caption{Legenda da Imagem}
    \label{fig:rotuloImagem}
\end{figure}
\newpage 

Não esqueça de referenciar suas tabelas no texto para que o Overleaf possa montar o índice de tabelas. O código da \autoref{tab:tabela} abaixo insere uma tabela que você pode gerar utilizando o gerador de tabelas. (\href{https://www.tablesgenerator.com/}{Clique aqui para Acessar o Gerador de Tabelas}. Esse passo a passo está melhor explicado no tutorial de Latex.\\

% comando para inserir tabelas
\begin{table}[!h]
\centering
\begin{tabular}{l|l|l}
\hline
\multicolumn{1}{c|}{\textbf{Tabela}} & \multicolumn{1}{c|}{\textbf{Título 1}} & \multicolumn{1}{c}{\textbf{Título 2}} \\ \hline
Assunto 1                            & A                                      & C                                     \\ \hline
Assunto 2                            & B                                      & D                                     \\ \hline
\end{tabular}
\caption{Tabela}\label{tab:tabela}
\end{table}


\chapter{Metodologia}  \label{cap:04}

A metodologia consiste num conjunto de etapas ordenadamente dispostas a serem executadas e que tenham por finalidade a investigação de fenômenos para a obtenção de conhecimentos. Basicamente, compõe-se de etapas dispostas de forma sistemática, obedecendo a uma forma sequencial. 

Sendo assim, para a elaboração de um Trabalho de Pesquisa em Tecnologia da Informação, é preciso responder detalhadamente as seguintes questões:\\

\textbf{Como se procederá a pesquisa?}

% enumerate é um comando que insere listas numeradas
\begin{enumerate}
    \item Qual será o tema da sua pesquisa? 
    \item Qual o espaço (local ou área) delimitado da pesquisa? 
    \item Qual é o pretende resolver?
    \item Qual será o tipo da sua pesquisa? Desenvolvimento de Software ou Pesquisa Bibliográfica?
    \item Se for realizar pesquisa Bibliográfica, qual será sua área de estudo? Que autores pretende abordar?
    \item Pretende realizar questionários ou entrevistas com pessoas da área?
    \item Se a pesquisa for de desenvolvimento de software, como será feita a análise de requisitos? Quem será consultado? 
    \item Como será construída a documentação do Sistema?
    \item Quais serão as tecnologias utilizadas para a documentação, desenvolvimento e testes de software?
    \item Como o sistema será construído?
    \item Como será implementado o sistema?
    \item Se for realizar testes de software, qual método pretende adotar?
\end{enumerate}

\chapter{Resultados Preliminares} \label{cap:05}

% Para a versão da banca final, essa seção muda de nome, se tornando Resultados e Discussão. Não apague esse comentário, para se lembrar de fazer essa alteração. Para a seção definitiva, você deve escrever sobre os resultados que obteve no decorrer de todo projeto.


Nessa seção você deve escrever brevemente sobre os resultados que já encontrou durante essa primeira fase da sua pesquisa.
Devemos lembrar que cada item é um resultado, seja texto escrito, diagramas e documentação de sistema, telas prontas ou qualquer artefato que você venha a produzir. Devemos lembrar também que resultados negativos devem ser apresentados, já que esses registros são importantes para que pesquisadores que usem seu trabalho como referencial não cometam o mesmo erro com a mesma abordagem.

Nessa seção você deve inserir tabelas, gráficos, imagens de dados coletados, partes do sistema e outros dados que já são considerados resultados. 



\chapter{Conclusões} \label{cap:06}

Após escrever os resultados encontrando até o presente momento, podemos fazer uma breve conclusão preliminar dos estudos. Para as conclusões preliminares é preciso dizer como está o andamento do trabalho, quais os objetivos estão sendo satisfeitos e uma breve descrição do que ainda é preciso fazer.

Lembre-se, nessa seção você apresenta citações diretas ou indiretas, não insire imagens ou gráficos, tabelas e mapas na conclusão.
\chapter{Cronograma}  \label{cap:07}

% Esse capítulo só se insere na Versão da Pré Banca. Na Versão da Defesa ele é retirado já que o cronograma deverá ter sido concluído.

\begin{table}[!h]
\begin{tabular}{|c|c|c|c|c|c|c|c|c|c|c|}
\hline
\multirow{2}{*}{\textbf{Descrição das Atividades}} & \multicolumn{10}{c|}{\textbf{Meses}}            \\ \cline{2-11} 
                                                   & 01 & 02 & 03 & 04 & 05 & 06 & 07 & 08 & 09 & 10 \\ \hline
Atividade 1                                        &    & $\bullet$ & $\bullet$  &    &    &    &    &    &    &    \\ \hline
Atividade 2                                        &    &    & $\bullet$  & $\bullet$  &    &    &    &    &    &    \\ \hline
Atividade 3                                        &    &    &    & $\bullet$  &    &    &    &    &    &    \\ \hline
Atividade 4                                        &    &    &    &    &    &    &    &    &    &    \\ \hline
Atividade 5                                        &    &    &    &    &    &    &    &    &    &    \\ \hline
Atividade 6                                        &    &    &    &    &    &    &    &    &    &    \\ \hline
\end{tabular}
\end{table}


No cronograma você irá inserir as atividades que irá realizar, bem como marcar com a bolinha nos meses que irá realizá-las. Como estamos no Latex, a edição dessas tabelas deve ser realizada em um programa que trabalhe com configuração de tabelas.
Para isso, você deve seguir os seguintes passos:

\begin{enumerate}
    \item Copie o código da tabela acima.
    \item Acesse o seguinte site: \href{https://www.tablesgenerator.com/latex_tables#}{Editor de Tabelas} ;
    \item Clique em "File";
    \item Clique em "From Latex Code";
    \item Cole o Texto da tabela copiada e clique em "Load";
\end{enumerate}

Esse procedimento irá gerar uma tabela como a que está acima. Basta editar conforme sua necessidade, alterando as atividades e o nome dos meses e posicionando a bolinha nos quadrados dos meses em que as atividades serão realizadas.

Posteriormente basta clicar em "Generate", copiar o código e substituir a tabela acima por ele.

% ----------------------------------------------
%         Referências bibliográficas
% ----------------------------------------------

%\bibliographystyle{plain}
\bibliography{referencias}

% ----------------------------------------------
%            ELEMENTOS PÓS-TEXTUAIS
% ----------------------------------------------
\postextual

% -----------------------------------------------
%                  Glossário
% -----------------------------------------------
%
% Consulte o manual da classe abntex2 para orientações sobre o glossário.
%%\glossary

% ------------------------------------------------
% Apêndices
% ------------------------------------------------
% Texto ou documento elaborado pelo autor, a fim de complementar sua argumentação, sem prejuízo da unidade nuclear do trabalho.

% ---
% Inicia os apêndices
% ---
\begin{apendicesenv}
	
	% Imprime uma página indicando o início dos apêndices
	%\partapendices
	
	% ----------------------------------------------------------
	\chapter{Título do Apêndice A}
	% ----------------------------------------------------------
	
	Texto do Apêndice A.
	
	.
		
	
\end{apendicesenv}
% ---
% ----------------------------------------------------------
% Anexos
% ----------------------------------------------------------
% Texto ou documento não elaborado pelo autor, que serve de fundamentação, comprovação e/ou ilustração.

% ---
% Iniciam os anexos
% ---
\begin{anexosenv}
	
	% Imprime uma página indicando o início dos anexos
	% \partanexos
	
	% ----------------------------------------------------------
	\chapter{Título do Anexo A}
	% ----------------------------------------------------------
	
	Texto do Anexo A.
	
	
\end{anexosenv}
%-----------------------------------------------------------
% ÍNDICE REMISSIVO
%-----------------------------------------------------------

%\phantompart

% \printindex

\end{document}
